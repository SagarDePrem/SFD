%% LyX 2.1.4 created this file.  For more info, see http://www.lyx.org/.
%% Do not edit unless you really know what you are doing.
\documentclass[english]{article}
\usepackage[T1]{fontenc}
\usepackage[latin9]{inputenc}
\usepackage{geometry}
\geometry{verbose,tmargin=2cm,bmargin=2cm,lmargin=2cm,rmargin=2cm,headheight=2.5cm,headsep=4cm,footskip=1cm}
\usepackage{fancyhdr}
\pagestyle{fancy}
\usepackage{textcomp}
\usepackage{pdfpages}
\usepackage{amstext}

\makeatletter

%%%%%%%%%%%%%%%%%%%%%%%%%%%%%% LyX specific LaTeX commands.
%% Because html converters don't know tabularnewline
\providecommand{\tabularnewline}{\\}
%% A simple dot to overcome graphicx limitations
\newcommand{\lyxdot}{.}


\makeatother

\usepackage{babel}
\begin{document}

\title{\textbf{AS 5540 Assignment 1}}


\author{Prem Sagar S - AE14B021}

\maketitle

\section*{Problem 1}

Given the initial position as 
\[
r_{0}=6837432.552\hat{i}+1868795.099\hat{j}+1455480.629\hat{k}m
\]


and inital velocity 
\[
v_{0}=-2294.079\hat{i}+6758.849\hat{j}+2049.468\hat{k}m/sec.
\]
 

Using these data in the orbit.m code, the position of the satellite
is calculated for 200 minutes. 

From the results (pages 2-4), the plot of the trajectory is elliptic.

minimum altitude = 852.503 km at\textbf{ 0.280533 h }and 875.06 km
at\textbf{ 1.12901 h}. 

The total time elapsed in moving from minimum altitude to maximum
altitude is half the time period.

Therefore, $T=2*60*(1.12901-0.0280533)=101.8174$ minutes.

The length of the major axis = sum of minimum and maximum altitude+earth
diameter

\[
2a=875.503+875.06+2*6378.137=14483.837km
\]


\[
a=7241.9185km
\]


The ellipse centre to earth's centre measures \textbf{ae}

\[
ae=7241.9185-6378.137-852.503
\]


which gives,
\[
e=0.0015574
\]


\begin{center}
\begin{tabular}{|c|c|}
\hline 
Position at $t_{f}$ & \tabularnewline
\hline 
Velocity at $t_{f}$ & \tabularnewline
\hline 
Type of trajectory & elliptic\tabularnewline
\hline 
semi-major axis  & 7241.9185 km\tabularnewline
\hline 
Eccentricity & 0.0015574\tabularnewline
\hline 
Time period & 101.8174 min\tabularnewline
\hline 
\end{tabular}
\par\end{center}

\includepdf[pages=1-3]{\string"prob 1 a\string"}

\includepdf[pages=1-3]{\string"prob 1 b\string"}


\section*{Problem 2}

From Kepler's third law,

\[
\frac{R_{mars}}{R_{earth}}=(\frac{T_{mars}}{T_{earth}})^{2/3}
\]
therefore the mean radius of mars is given by,

\[
R_{mars}=(\frac{687}{365.26})^{3/2}*149.59787*10^{6}=385.884508*10^{6}km
\]
Orbital speed of Earth,

\[
\omega_{earth}=\frac{2\pi}{T_{earth}}=\frac{2\pi}{365.26*24*60*60}=1.990967*10^{-7}rad/s
\]
Orbital speed of Mars,

\[
\omega_{mars}=\frac{2\pi}{T_{mars}}=\frac{2\pi}{687*24*60*60}=1.058545155*10^{-7}rad/s
\]
Assuming the mean radius, gravitational force of Sun on Earth, ($\hat{r_{se}}$
is directed from Earth to sun and vice versa)

\[
F_{es}=-\frac{Gm_{earth}m_{sun}}{R_{earth}^{2}}\hat{r_{es}}=-\frac{398600km^{3}/s^{2}*1.989\text{\texttimes}10^{30}kg}{(149.59787*10^{6})^{2}km^{2}}\hat{r_{es}}=-3.542593*10^{19}N\hat{r_{es}}
\]
From Newton's third law, force on sun due to earth is 

\[
F_{se}=-F_{es}=+3.542593*10^{19}N\hat{r_{es}}
\]
Assuming the mean radius, gravitational force of Sun on Mars, ($\hat{r_{ME}}$
is directed from Mars to sun and vice versa)

\[
F_{ms}=-\frac{Gm_{mars}m_{sun}}{R_{mars}^{2}}\hat{r_{ms}}=-\frac{42828km^{3}/s^{2}*1.989\text{\texttimes}10^{30}kg}{(385.884508*10^{6})^{2}km^{2}}\hat{r_{ms}}=-5.720683*10^{17}N\hat{r_{ms}}
\]
From Newton's third law, force on Sun due to Mars is 

\[
F_{sm}=-F_{ms}=+5.720683*10^{17}N\hat{r_{ms}}
\]



\section*{Problem 3}

Given that the satellite is moving in an elliptical orbit, the specific
energy is given by,

\[
\epsilon=-\frac{\mu}{2a}
\]
Perigee radius is given by \textbf{$r_{p}=$a(1-e)}

It can also be calculated as, 
\[
r_{p}=\frac{h^{2}/\mu}{1+ecos(0)}=\frac{h^{2}/\mu}{1+e}
\]
\[
a(1-e)=\frac{h^{2}/\mu}{1+e}
\]
Thus, the specific angular momentum can be obtained as,

\[
h=\sqrt{\mu a(1-e^{2})}
\]
The code enclosed has assumed satellite mass is negligible compared
to Earth's mass.

\[
\epsilon=-16.1320km^{2}/s^{2}
\]


\[
h=6.726037*10^{4}km^{2}/s
\]



\subsection*{Radial velocity}

\[
v_{r}=\dot{r}=\frac{d}{dt}(\frac{h^{2}/\mu}{1+ecos\theta})=\frac{h^{2}/\mu}{-(1+ecos\theta)^{2}}(-esin\theta)\dot{\theta}
\]
\[
=\frac{h^{4}/\mu^{2}}{(1+ecos\theta)^{2}}(esin\theta)\dot{\theta}*\frac{1}{h^{2}/\mu}=r^{2}\dot{\theta}*esin\theta*\frac{\mu}{h^{2}}
\]
\[
=h*esin\theta*\frac{\mu}{h^{2}}=\frac{\mu esin\theta}{h}
\]



\subsection*{Normal velocity}

The angular momentum is given by,

\[
h=\vec{r}\times\vec{v}=\vec{r}\times(\vec{v_{\bot}}+\vec{v_{r}})=\vec{r}\times\vec{v}_{\bot}+\vec{r}\times\vec{v}_{r}=\vec{r}\times\vec{v_{\bot}}+0
\]
where, $\vec{v_{\bot}}$and $\vec{v_{r}}$ are the normal and radial
velocities respectively.

This simplifies to,

\[
h=rv_{\bot}
\]


\[
v_{\bot}=\frac{h}{r}=\mu\frac{1+ecos\theta}{h}
\]


\[
\]


\includepdf[pages=1-3]{\string"prob 3 a\string"}

\includepdf[pages=1-3]{\string"prob 3 b\string"}


\section*{Problem 4}

Given the meteoroid is at 402000km with $\theta=150^{0}$and velocity
v=2.23km/s, semi major axis can be computed from the specific energy
relation.

\[
\epsilon=\frac{\mu}{2a}=\frac{v^{2}}{2}-\frac{\mu}{r}
\]
\[
a=\sqrt{\frac{\mu}{2(\frac{v^{2}}{2}-\frac{\mu}{r})}}
\]


From the previous problem, the total speed is obtained as the square
of normal and radial velocities,

\[
v^{2}=v_{r}^{2}+v_{\bot}^{2}=[\frac{\mu esin\theta}{h}]^{2}+[\frac{\mu(1+ecos\theta)}{h}]^{2}
\]
\[
v^{2}=\frac{\mu^{2}}{h^{2}}[e^{2}sin^{2}\theta+1+e^{2}cos^{2}\theta+2ecos\theta]
\]
\[
v^{2}=\frac{\mu}{r(1+ecos\theta)}[e^{2}+1+2ecos\theta]
\]
This can be simplified to a quadratic in \textbf{e,}

\[
e^{2}+(2-\frac{v^{2}r}{\mu})cos\theta e+1-\frac{v^{2}r}{\mu}=0
\]
Since the meteoroid is known to be in a hyperbolic path, the positive
root greater than 1 is taken as the eccentricity.

The radius at closest approach= $r_{p}=a(e-1)$.

The velocity at closest approach can be computed from the energy relation,

\[
\epsilon=\frac{\mu}{2a}=\frac{v_{p}^{2}}{2}-\frac{\mu}{r_{p}}
\]


\[
v_{p}=\sqrt{2(\frac{\mu}{2a}+\frac{\mu}{r_{p}})}
\]
Having known the semi major axis and eccentricity, the trajectory
can be determined from the equation of hyperbola given by,
\[
\frac{x^{2}}{a^{2}}-\frac{y^{2}}{a^{2}(1-e^{2})}=1
\]


\includepdf[pages=1-2]{\string"problem 4\string"}


\section*{Problem 5}

Given mean anomaly $M_{e}=$and eccentricity $e=$, the eccentric
anomaly can be obtained from the Kepler's equation

\[
M_{e}=E-esinE
\]
The difficulty in obtaining an analytical solution for E gives way
to using Newton Raphson method to obtain \textbf{E.}

\[
E=2.772571radians
\]


\includepdf[pages=1-2]{\string"problem 5\string"}
\end{document}
